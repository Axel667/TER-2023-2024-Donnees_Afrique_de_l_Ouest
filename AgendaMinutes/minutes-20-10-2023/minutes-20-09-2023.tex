\documentclass{article}

\usepackage[utf8]{inputenc}
\usepackage[T1]{fontenc}
\usepackage[francais]{babel}
\usepackage{titling}
\setlength{\droptitle}{-3cm}

\title{Objet : Deuxième réunion avec nos encadrants}
\author{Axel CAROT}
\date{\today}

\begin{document}
\maketitle

\section{Compte rendu de la documentation}
Après avoir présenté notre premier rapport à nos commanditaires, nous avons, à tour de rôle, résumé notre documentation. Les résumés sont disponibles dans le dossier "Documentation" du dépôt Github. Cela nous a permis de comprendre la documentation en détail, ainsi que de partager avec les autres membres du groupe ce que nous avions appris. Nous avons également pu poser nos différentes questions, bien que nous ne soyons pas encore entrés dans les détails techniques, nous nous sommes principalement concentrés sur les aspects théoriques pour le moment. \\

Discuter de la documentation avec les commanditaires nous a permis de mieux appréhender des concepts abstraits tels que la vectorisation des mots dans l'espace et son utilité. Cela nous a également permis de clarifier des points que nous n'avions pas réussi à saisir grâce à leurs explications. Par exemple, nous n'avions pas compris comment avait été établi l'indicateur de sécurité alimentaire basé sur le texte (TXT-FS).

\section{À faire avant la prochaine réunion}

\begin{itemize}
    \item Créer un canal Discord avec les commanditaires.
    \item Résumer les étapes principales du pipeline.
    \item Commencer à réfléchir à des modifications dans les étapes du pipeline : enrichir les données d’entrée, compléter le lexique des mots-clés, envisager le remplacement du topic modeling par des techniques plus récentes telles que Berth (Large language model), Camemberth, Floberth.
    \item Examiner les modèles et essayer de les mettre en marche.
    \item Réfléchir à la manière d'enrichir les données d'entrée à partir de la radio.
\end{itemize}

\end{document}
