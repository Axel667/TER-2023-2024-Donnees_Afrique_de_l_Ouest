\documentclass{article}

\usepackage[utf8]{inputenc}
\usepackage[T1]{fontenc}
\usepackage[francais]{babel}
\usepackage{titling}
\setlength{\droptitle}{-3cm}
\usepackage{graphicx} 

\title{Planification et gestion de projet}
\author{CAROT Axel, ARISOY Ivan Can, \\ DARDE Guilhem, NDJINGA NDJINGA Anta Claude}
\date{\today}

\begin{document}
\maketitle

\section{Élection du «Team Leader»}
Nous avons décidé d'élire ... en tant que TEAM Leader pour les raisons suivantes. Son rôle sera de :
\begin{itemize}
    \item coordonner les efforts de tous les membres de l’équipe
    \item aider à répartir les différentes tâches et activités en fonction des forces, compétences et intérêts de chacun
    \item organiser, de façon collégiale, les rôles au sein de l’équipe
    \item fluidifier la circulation d’information entre les membres
    \item faire un suivi de l’avancement des sous-tâches, et identifier en amont les
    \item difficultés rencontrées I motiver les troupes
\end{itemize}

\section{Planification des tâches et stratégie}

\subsection{Sélection d'un outil de gestion de projet : Trello}
Expliquer pourquoi on utilise Trello. Trello nous permet d'avoir des visualisations graphiques pour une meilleure représentation visuelle de notre planification. 

\subsection{Planification des tâches}
Notre organisation actuelle repose principalement sur la répartition des tâches en vue des échéances liées aux rapports à rendre au cours du semestre. 

mettre une image du trello

\subsection{Stratégie}
Mettre en place une stratégie bien définie, avec des jalons clairs et mesurables, et des échéances réalistes

faire un diagramme de gantt

\subsection{Google Agenda}
L'utilisation de Google Agenda nous permet de synchroniser nos rendez-vous, de fixer les dates de nos débriefings et de suivre nos diverses échéances. 

mettre une image

\end{document}
